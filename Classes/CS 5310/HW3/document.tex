\documentclass[11pt,twoside]{IEEEtran}

%opening
\title{How Architects See Non-Functional Requirements: Beware of Modifiability}
\author{Paper by Eltjo R. Poort, Nick Martens, Inge van de Weerd, and Hans van Vliet\\Reviewed by Ryan Darras}

\begin{document}

\maketitle

\section{Suggestion for acceptance}
Mildly reject. Easy read, but I don't think they built this paper with strong enough data to come up with an accurate conclusion. For all we know, their conclusion is very far from reality.

\section{Summary}
The authors explore the key findings of a survey about dealing with non-functional requirements among architects. They claim that non-functional requirements do not adversely affect the projects success as long as the architects are aware of the importance, with one exception: modifiability is detrimental to project success even when the architect is aware of it. The authors conclude that  modifiability deserves more attention and they claim that in general it is quantified and verified less than other non-functional requirements. They also claim that projects that applied non-functional requirement verification techniques early in development were more successful on average than projects that did not.

By determining if certain types of non-functional requirements have a relationship with project success, the authors want to show how important it is to consider modifiability when eliciting project requirements to ensure a high quality result. They conclude that modifiability has a huge play on customer satisfaction when the modifiability non-functional requirement is business critical.

\section{Positive points}
Overall, this paper is an easy read. I've done a bit in this area but never really anything documented or logged... just discussion, but the authors write this in such a way that I can easily read along and understand their points.

Fig 2 shows two example questions from their survey. Both of which I think are very good questions. However, I would have preferred them to have included every question in this paper.

\section{Negative points}
Fig 1. What does this figure offer to the paper? It doesn't really contain anything that couldn't be said in a few words.

They collected 133 responses to a survey with 23 questions, of which they eliminated 1 for being a duplicate and 51 for incomplete responses giving only 81 actual results which is not near enough to be held accountable and actually realistic. Not to mention the fact that if 51 were incomplete, what kind of "professionals" were they surveying. For all we know, the 81 results that were used as reference in this paper were bullshitted in 5-10 minutes, or perhaps they were falsified to make the reporting company look better.

\section{Potential future work}
The same kind of study can be done with every other non-functional requirement. This paper focused explicitly on modifiability, but individual papers can be done which focus on every non-functional requirement. Preferably with more research on each.

Given the previous segment of future work is accomplished, I would also like to see a scenario where a survey is done of all of these non-functional requirements to see which ones play big roles and where. It would be nice to discover a cookie cutter tool to figure out, based on the type of application you are building, which non-functional requirements should you 100\% make sure to elicit from the customer.

I would also just like to see this paper reconsidered with a better format for the surveys. Instead of just giving the architects a survey that they fill out whenever and however, I think shadowing these architects during the requirements elicitation phase and concluding success of the project when it is finished. Sure, that is a ton of work to put into a paper, but you know those results would be sound!

\end{document}
