\documentclass[10pt,twoside]{IEEEtran}

%opening
\title{How Cloud Providers Elicit Consumer Requirements: An Exploratory Study of\\ Nineteen Companies}
\author{Paper by Irina Todoran, Norbert Seyff, Martin Glinz\\Reviewed by Ryan Darras}

\begin{document}

\maketitle

\section{Suggestion for acceptance}
Strong Reject due to contradictions, small sample size, and general ambiguity.

\section{Summary}
This paper explores the requirements elicitation strategies differences between nineteen cloud service providers. They conducted interviews averaging 70 minutes with only architects, project managers, consultants and software engineers who had direct contact with requirements identification activities. They repeatedly fall back on the conception that traditional software requirements elicitation allows for in-person interviews, whereas cloud service providers generally have to conduct interviews via phone or a chat service like skype. The authors discovered that traditional approaches and prototyping as still the most frequently used elicitation techniques.

\section{Positive points}
``When we discussed with two employees of the same company, the interviews were individual and usually targeted complementary topics" - I like this, considering you are likely talking about something very similar so you can get two viewpoints on it, but you also get feedback for more than what you could get in just one interview. \\

``For example, if a provider used interviews in the past and it recently started to deliver cloud services along with its off-the-shelf software, it tries to apply the same methods also for the new scenario'' - Great use of an example here. I wish they had more like this throughout the paper.


\section{Negative points}
The paper didn't do a very good job describing exactly what software/hardware vs cloud services were being compared/contrasted. They explain how cloud services are still in their infancy so RE techniques have not been developed/adapted quite yet. From my perspective, using cloud hosting as an example, the differences are almost non-existent when comparing to traditional in-house server stacks. You still have to consider the power behind the server, the expected load, the hdd capacity, etc. One thing that is new is dynamic scalability with cloud hosting. Are we talking about 3rd party cloud services, server hosting.. what?\\

``We first contacted employees of cloud provider companies we personally knew'' - I completely understand how this is often times very effective, but for all we know these cloud provider companies" are simple basement projects. \\

The following two quotes directly contradict each other.

``The results of this exploratory study show that, whereas a few cloud providers try to implement and adapt traditional methods, the large majority uses ad-hoc
approaches for identifying consumer needs.'' 

``We found that a significant number of the elicitation methods implemented by cloud providers are well-known approaches used by traditional software suppliers.'' \\

``Since the study was directly conducted with practitioners, we consider the results relevant for both industry and academia.'' - Interviewing 19 companies doesn't fit the bill for being ``relevant for both industry and academia''. If they interviewed 30+ cloud service providers as well as 50+ cloud service consumers then I would consider it relevant.

\section{Potential future work}
Cloud computing is still in its infancy, as stated in the paper. This means that a study of existing companies might not be the best option. However, this opens the possibility of conducting research involving different strategies to conduct requirements elicitation in regards to cloud service providers. This offers an opportunity to gather data from traditional requirements elicitation techniques in the cloud service realm, as well as add brand new state-of-the-art techniques that could be used by cloud service providers and become a staple in this domain of research.\\

\end{document}
